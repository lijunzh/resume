%-------------------------------------------------------------------------------
%  Copyright (c) 2025 Lijun Zhu
%  Licensed under Creative Commons Attribution-ShareAlike 4.0 International (CC BY-SA 4.0)
%  https://creativecommons.org/licenses/by-sa/4.0/
%-------------------------------------------------------------------------------

%%%%%%%%%%%%%%%%%%%%%%%%%%%%%%%%%%%%%%%%%
% "ModernCV" CV and Cover Letter
% LaTeX Template
% Version 1.1 (9/12/12)
%
% This template has been downloaded from:
% http://www.LaTeXTemplates.com
%
% Original author:
% Xavier Danaux (xdanaux@gmail.com)
%
% License:
% CC BY-NC-SA 3.0 (http://creativecommons.org/licenses/by-nc-sa/3.0/)
%
% Important note:
% This template requires the moderncv.cls and .sty files to be in the same
% directory as this .tex file. These files provide the resume style and themes
% used for structuring the document.
%
%%%%%%%%%%%%%%%%%%%%%%%%%%%%%%%%%%%%%%%%%

%----------------------------------------------------------------------------------------
%	PACKAGES AND OTHER DOCUMENT CONFIGURATIONS
%----------------------------------------------------------------------------------------

% Font sizes: 10, 11, or 12; paper sizes: a4paper, letterpaper, a5paper, 
% legalpaper, executivepaper or landscape; font families: sans or roman
\documentclass[11pt,a4paper,sans]{moderncv}

% CV theme - options include: 'casual' (default), 'classic', 'oldstyle' and 'banking'
\moderncvstyle{casual}
% CV color - options include: 'blue' (default), 'orange', 'green', 'red', 
% 'purple', 'grey' and 'black'
\moderncvcolor{grey}

\usepackage{lipsum} % Used for inserting dummy 'Lorem ipsum' text into the template

\usepackage{geometry} % Reduce document margins
\geometry{left=1.4cm, top=1.8cm, right=1.4cm, bottom=1.8cm, footskip=.5cm}
%\setlength{\hintscolumnwidth}{3cm} % Uncomment to change the width of the dates column
% For the 'classic' style, uncomment to adjust the width of the space 
% allocated to your name
%\setlength{\makecvtitlenamewidth}{10cm}

%----------------------------------------------------------------------------------------
%	NAME AND CONTACT INFORMATION SECTION
%----------------------------------------------------------------------------------------

\firstname{Dr. Lijun} % Your first name
\familyname{Zhu} % Your last name

% All information in this block is optional, comment out any lines you don't need
\title{Teaching Statement}
\address{3505 Leighton Ridge Dr.}{Plano, TX 75025}
\mobile{(404) 545-2619}
%\phone{(000) 111 1112}
%\fax{(000) 111 1113}
\email{gatechzhu@gmail.com}
% The first argument is the url for the clickable link, the second argument is 
% the url displayed in the template - this allows special characters to be 
% displayed such as the tilde in this example
%\homepage{staff.org.edu/~jsmith}{staff.org.edu/$\sim$jsmith}
%\extrainfo{additional information}
% The first bracket is the picture height, the second is the thickness of the 
% frame around the picture (0pt for no frame)
%\photo[70pt][0.4pt]{pictures/picture}
%\quote{"A witty and playful quotation" - John Smith}

%----------------------------------------------------------------------------------------

\begin{document}
\makecvtitle % Print the CV title

\section{Teaching Philosophy}
As an educator and leader in data science, I believe teaching is not just about imparting knowledge, but about inspiring students to discover their passions and approach the world with curiosity. My goal is to foster a learning environment that encourages critical thinking, nurtures a love of learning, and prepares students to meet future challenges.

{\hskip 2em}\textbf{Student-Centered Learning}: I prioritize an approach where students’ needs and perspectives shape the course content. This includes interactive discussions, collaborative projects, and real-world applications to make learning relevant and engaging.

{\hskip 2em}\textbf{Inclusivity and Diversity}: I recognize and value the diverse backgrounds of my students, striving to create an inclusive classroom where every student feels respected and empowered to contribute their unique insights.

{\hskip 2em}\textbf{Critical Thinking and Problem Solving}: I emphasize the development of critical thinking skills, encouraging students to question, analyze, and synthesize information—preparing them for complex problem-solving in their careers.

\section{Teaching Methods}
Throughout my teaching experience, I have employed a diverse set of tools to engage students, continuously improved my course content and teaching approaches based on feedback, and supported students creatively in achieving their learning goals.

{\hskip 2em}\textbf{Blended Learning}: I incorporate a mix of traditional lectures, digital resources, and hands-on activities to address various learning styles. For example, I organized a monthly “lunch-and-learn” lecture series in my current and previous organizations, educating both junior associates and senior leaders on key topics in modern artificial intelligence. In addition to introducing foundational theory and concepts, I leverage visualization, coding practice, and mini-hackathons on small proof-of-concept projects to engage participants. Surveys after these sessions consistently show that hands-on coding is the most effective way to quickly grasp complex and unfamiliar machine learning concepts.

{\hskip 2em}\textbf{Feedback and Continuous Improvement}: Regular feedback from students is essential. I use it to adapt my teaching strategies and ensure I am meeting their learning needs. As a teaching assistant for an undergraduate digital design course at Georgia Tech, I regularly interviewed students at the end of each semester to understand what they appreciated and what could be improved. I discussed this feedback with fellow TAs and professors, using it to refine course materials for future semesters. This practice helped bridge the gap between what educators aim to teach and what students actually receive. I later applied these insights in management roles, finding them invaluable for leading technical teams. When guest lecturing at the University of Texas at Dallas, I actively sought feedback from Professor James Zhang and adjusted my lectures accordingly, resulting in improved student engagement over time.

{\hskip 2em}\textbf{Mentorship and Support}: Beyond the classroom, I am committed to mentoring students in their academic and professional pursuits, offering guidance as they navigate their career paths. I participated in the Vertical Integrated Program (VIP) for several years, helping to build an adaptive mentoring system for an undergraduate signal processing course. We designed an interactive quiz app that provided students with questions of adaptive difficulty based on their previous performance, response time, and personal preferences—offering a customized learning experience at their own pace. I also host personal mentoring sessions and office hours to answer questions and gather first-hand feedback, continuously improving the learning experience.

\section{Future Goals}
I aspire to adapt my current course materials for undergraduate and graduate courses in data science and business analytics, enriched with real-world industrial examples. I am also interested in developing focused graduate courses on topics highly relevant to today’s business environment, such as generative AI (LLMs, stable diffusion, etc.) and graph neural networks (GNNs). I believe my extensive industry background and commitment to education will benefit the learning community at the University of Texas at Dallas. I look forward to challenging myself and my students, preparing them to become thoughtful, innovative, and responsible global citizens.

\end{document}
