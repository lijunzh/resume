%-------------------------------------------------------------------------------
%  Copyright (c) 2025 Lijun Zhu
%  Licensed under Creative Commons Attribution-ShareAlike 4.0 International (CC BY-SA 4.0)
%  https://creativecommons.org/licenses/by-sa/4.0/
%-------------------------------------------------------------------------------

%%%%%%%%%%%%%%%%%%%%%%%%%%%%%%%%%%%%%%%%%
% "ModernCV" CV and Cover Letter
% LaTeX Template
% Version 1.1 (9/12/12)
%
% This template has been downloaded from:
% http://www.LaTeXTemplates.com
%
% Original author:
% Xavier Danaux (xdanaux@gmail.com)
%
% License:
% CC BY-NC-SA 3.0 (http://creativecommons.org/licenses/by-nc-sa/3.0/)
%
% Important note:
% This template requires the moderncv.cls and .sty files to be in the same
% directory as this .tex file. These files provide the resume style and themes
% used for structuring the document.
%
%%%%%%%%%%%%%%%%%%%%%%%%%%%%%%%%%%%%%%%%%

%----------------------------------------------------------------------------------------
%	PACKAGES AND OTHER DOCUMENT CONFIGURATIONS
%----------------------------------------------------------------------------------------

% Font sizes: 10, 11, or 12; paper sizes: a4paper, letterpaper, a5paper, 
% legalpaper, executivepaper or landscape; font families: sans or roman
\documentclass[11pt,a4paper,sans]{moderncv}

% CV theme - options include: 'casual' (default), 'classic', 'oldstyle' and 'banking'
\moderncvstyle{casual}
% CV color - options include: 'blue' (default), 'orange', 'green', 'red', 
% 'purple', 'grey' and 'black'
\moderncvcolor{grey}

\usepackage{lipsum} % Used for inserting dummy 'Lorem ipsum' text into the template

\usepackage{geometry} % Reduce document margins
\geometry{left=1.4cm, top=1.8cm, right=1.4cm, bottom=1.8cm, footskip=.5cm}
%\setlength{\hintscolumnwidth}{3cm} % Uncomment to change the width of the dates column
% For the 'classic' style, uncomment to adjust the width of the space 
% allocated to your name
%\setlength{\makecvtitlenamewidth}{10cm}

%----------------------------------------------------------------------------------------
%	NAME AND CONTACT INFORMATION SECTION
%----------------------------------------------------------------------------------------

\firstname{Dr. Lijun} % Your first name
\familyname{Zhu} % Your last name

% All information in this block is optional, comment out any lines you don't need
\title{Teaching Statement}
\address{3505 Leighton Ridge Dr.}{Plano, TX 75025}
\mobile{(404) 545-2619}
%\phone{(000) 111 1112}
%\fax{(000) 111 1113}
\email{gatechzhu@gmail.com}
% The first argument is the url for the clickable link, the second argument is 
% the url displayed in the template - this allows special characters to be 
% displayed such as the tilde in this example
%\homepage{staff.org.edu/~jsmith}{staff.org.edu/$\sim$jsmith}
%\extrainfo{additional information}
% The first bracket is the picture height, the second is the thickness of the 
% frame around the picture (0pt for no frame)
%\photo[70pt][0.4pt]{pictures/picture}
%\quote{"A witty and playful quotation" - John Smith}

%----------------------------------------------------------------------------------------

\begin{document}
\makecvtitle % Print the CV title

\section{Teaching Philosophy}
As an aspiring leader in data science, my teaching philosophy is rooted in the belief 
that education is not just about imparting knowledge, but about inspiring students to 
discover their passions and explore the world with an inquisitive mind.
My goal is to create a learning environment that encourages critical thinking, fosters 
a love for learning, and prepares students for the challenges of the future.

{\hskip 2em}\textbf{Student-Centric Learning}: I believe in an approach where student 
needs and perspectives shape the course content.
This involves interactive discussions, collaborative projects, and real-world applications 
to make learning relevant and engaging.

{\hskip 2em}\textbf{Inclusivity and Diversity}: Recognizing the diverse backgrounds of my 
students, I strive to create an inclusive classroom where every student feels valued and 
has the opportunity to contribute their unique insights.

{\hskip 2em}\textbf{Critical Thinking and Problem Solving}: I emphasize developing critical 
thinking skills, encouraging students to question, analyze, and synthesize information, 
preparing them for complex problem-solving in their future careers.

\section{Teaching Method}
Through out the years of my teaching experience, I engaged a diverse collections of tools 
to interact with my students, continuously improved my course content and teaching 
approaches via feedbacks, and helping students in a creative way to achieve their 
learning goals.

{\hskip 2em}\textbf{Blended Learning}: I incorporate a mix of traditional lectures, digital 
resources, and hands-on activities to cater to various learning styles.
I organized a monthly ``lunch-and-learn'' lecture series in my current and previous 
organizations, in which I try to educate junior associates as well as senior leaders 
on popular topics in modern artificial intelligence fields.
In addition to a brief introduction on the basic theory and concepts, I heavily leveraged 
visualization, coding practice, and mini-hackathon on small proof-of-concept (POC) 
projects to engage my audience in the one-hour sessions.
According to the survey conducted after the sessions, my audience found the hands-on 
coding session being the most effective way to quickly understand the complex and 
unfimilar concepts in machine learning.

{\hskip 2em}\textbf{Feedback and Continuous Improvement}: Regular feedback from students is 
crucial. I use it to adapt my teaching strategies and ensure that I am meeting their 
learning needs.
When I was the teaching assistant (TA) for an undergraduate digital design course for 
multiple semesters in Georgia Tech, I regularly interviewed my students at the end of 
the semesters to understand what they liked and what I hate about me as their TA.
I also discussed my feedbacks from students with my peer TAs and course professors 
during the break and try to address the concerns and prepare the course material better 
for the coming semesters. 
This practice helped me bridge the gap between what we, as educator, want to teach and 
what students received on their ends.
I later bring these understanding to my management roles and found them also very helpful 
in leading technical teams.
When I was giving guest lectures in University of Texas at Dallas hosted by Prof. James 
Zhang, I repeated seek feedbacks from Prof. Zhang and adjust my lecture notes accordingly.
I found that student engagement was improved lecture over lecture after the improvement 
from those feedbacks.

{\hskip 2em}\textbf{Mentorship and Support}: Beyond the classroom, I am committed to 
mentoring students in their academic and professional pursuits, offering guidance and 
support as they navigate their career paths.
I participated in vertical integrated program (VIP) for many years where I helped to 
build an adaptive mentoring system for undergraduate signal processing course. 
In the system, we designed an interactive quiz app that students will be given questions 
with adaptive difficulties based on their previous accuracies, response time, and their 
own difficulty preference. 
We tried to gave students an customizable learning experience that will support their 
learning experience in their own pace back in early 2010s. 
I also host personal mentoring sessions / office hours to answer questions from students 
and get the first-hand experience / feedbacks from them to continuously improve the 
system.

\pagebreak

\section{Future Goals}
I would love to adapt my current course material to undergraduate or graduate courses for 
students interested in data science and business analytics with rick industrial context 
and examples. 
I am also interested in more focused graduate courses on a specific topics that is most 
relevant to the current business environment, such as generative AI (LLM, stable 
diffusion, etc.) and graph neural networks (GNN).
I believe that my rich industry backgrounds and commitment to education will be able to 
benefit the community of learners in University of Texas at Dallas.
I would love to challenge myself as well as my students preparing them to be thoughtful, 
innovative, and responsible global citizens.

\end{document}
